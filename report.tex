\documentclass[11pt]{article}

\usepackage{amsmath}
\usepackage{hyperref}
\usepackage{graphicx}

\begin{document}


\begin{titlepage}
    \centering
    \includegraphics[width=1\textwidth]{images/logo_Uni.png}\par\vspace{1cm}
    {\scshape\Large Report WP1 - Vegetation \par}
    \vspace{1cm}
    {\scshape\large Giulio CARPI LAPI, Pierre-Antoine SENGER\par}
    \vspace{1cm}
    {\large \today\par}
\end{titlepage}
 
\tableofcontents % Table of Contents

\newpage % Start a new page after the TOC

\section{Introduction}

A team of researchers is currently developing a tool for building a 3D geometric model of an urban area (neighborhood, city, etc.).
This representation then enables them to create a digital thermal and energy simulation model of an urban area.
For the moment, they can reconstruct the geometry of buildings and land using information available in online databases such as OpenStreetMap, Mapbox, etc.
Vegetation (trees in particular) can have a significant influence on the model. We need to create these objects in the 3D model.

\section{Roadmap}
\begin{itemize}
	\item 
\end{itemize}

\begin{thebibliography}{9}
    \bibitem{osm-wiki} \url{https://wiki.openstreetmap.org/wiki/Overpass_API}
	\bibitem{osm-queries} \url{https://osm-queries.ldodds.com/tutorial}
	\bibitem{osm-learnoverpass} \url{https://osmlab.github.io/learnoverpass//en/}

\end{thebibliography}


\end{document}