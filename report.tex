\documentclass[12pt]{article}

\usepackage{amsmath}
\usepackage{hyperref}
\usepackage{graphicx}
\usepackage{float}
\usepackage{caption}
\usepackage{listings}
\usepackage{xcolor}

% Define a new environment for C++ code
\lstnewenvironment{cpp}
    {\lstset{
        language=C++,
        basicstyle=\small\ttfamily,
        keywordstyle=\color{blue},
        commentstyle=\color{green!40!black},
        stringstyle=\color{orange},
        numbers=left,
        numberstyle=\tiny,
        numbersep=5pt,
        breaklines=true,
        showstringspaces=false,
        frame=tb,
        captionpos=b
    }}
    {}

% Define a new environment for JSON code
\lstdefinelanguage{json}{
    basicstyle=\normalfont\ttfamily,
    numbers=left,
    numberstyle=\scriptsize,
    stepnumber=1,
    numbersep=8pt,
    showstringspaces=false,
    breaklines=true,
    frame=lines,
    backgroundcolor=\color{background},
    stringstyle=\color{stringstyle},
    keywordstyle=\color{keywordstyle},
    commentstyle=\color{commentstyle},
    morestring=[b]",
    morestring=[d]',
    morecomment=[l]{//},
    morecomment=[s]{/*}{*/},
    morekeywords={false,true,null},
    literate=
     *{0}{{{\color{numberstyle}0}}}{1}
      {1}{{{\color{numberstyle}1}}}{1}
      {2}{{{\color{numberstyle}2}}}{1}
      {3}{{{\color{numberstyle}3}}}{1}
      {4}{{{\color{numberstyle}4}}}{1}
      {5}{{{\color{numberstyle}5}}}{1}
      {6}{{{\color{numberstyle}6}}}{1}
      {7}{{{\color{numberstyle}7}}}{1}
      {8}{{{\color{numberstyle}8}}}{1}
      {9}{{{\color{numberstyle}9}}}{1}
}

\definecolor{numberstyle}{rgb}{0.17, 0.57, 0.68}
\definecolor{stringstyle}{rgb}{0.5, 0.3, 0.2}
\definecolor{keywordstyle}{rgb}{0.13, 0.29, 0.53}
\definecolor{commentstyle}{rgb}{0.25, 0.5, 0.37}
\definecolor{background}{rgb}{0.9, 0.9, 0.9}


\setlength{\parskip}{1em}

\begin{document}

\begin{titlepage}
    \centering
    \includegraphics[width=0.5\textwidth]{images/logo_ufr.png}\par\vspace{1cm}
    \vspace{1.5cm}
    {\huge\bfseries ExaMA WP1 - Vegetation\par}
    \vspace{2cm}
    {\Large Giulio Carpi Lapi, Pierre-Antoine Senger\par}
    \vfill
    supervised by\par
    Pierre Alliez and Vincent Chabannes

    \vfill

% Bottom of the page
    {\large Date: \today\par}
\end{titlepage}

\tableofcontents
\newpage

\section{Abstract}
Urban areas are intricate systems influenced by various factors,
including vegetation, particularly trees, which play a vital role in shaping
microclimates, energy consumption, and overall livability. This project focuses
on integrating vegetation, specifically trees, into 3D geometric models of
urban environments to enhance the accuracy and realism of thermal and energy
simulations. Leveraging data from sources like OpenStreetMap, the project aims
to identify tree positions and attributes within urban landscapes and generate
a library of 3D tree models for integration into terrain meshes. Supervised by
experts from Inria and the University of Strasbourg, the project follows a
roadmap with defined milestones to deliver versions V0, V1, and V2 by specified
deadlines. Through this endeavor, the project seeks to provide valuable
insights for urban planners, architects, and policymakers, ultimately
contributing to the development of more sustainable and resilient cities.

For the \textbf{V1} version of the report add:

\begin{itemize}
    \item Methodology
    \item Results
    \item Conclusion
\end{itemize}

\newpage

\section{Introduction}
Urban areas are complex environments where various factors interact to influence 
microclimates and energy consumption. Among these factors, 
vegetation, particularly trees, plays a crucial role in shaping the urban landscape 
and its environmental characteristics. Trees provide shade, mitigate heat and can
reduce air pollution. Their strategic placement and characteristics can
significantly impact the thermal and energy dynamics of urban areas.

\begin{figure}[H]
    \centering
    \includegraphics[width=0.5\textwidth]{images/TreeShade.png}
    \captionsetup{font={scriptsize}}
    \caption{Urban trees \cite{img:TreeShade}}
\end{figure}

\begin{figure}[H]
    \centering
    \includegraphics[width=0.5\textwidth]{images/heat_street.png}
    \captionsetup{font={scriptsize}}
    \caption{Thermal image of a street in the city. \cite{img:street_thermography}}
\end{figure}

In recent years, advancements in computational modeling have facilitated the 
development of sophisticated tools for simulating the thermal and energy performance 
of urban environments. These tools rely on 3D geometric models that accurately 
represent the built environment, including buildings, roads, and terrain.

\begin{figure}[H]
    \centering
    \includegraphics[width=0.5\textwidth]{images/thermographie-aerienne.jpg}
    \captionsetup{font={scriptsize}}
    \caption{Thermal image of a city \cite{img:aerialview}}
\end{figure}

In this context, this project aims to develop a methodology for integrating 
vegetation, particularly trees, into 3D geometric models of urban areas.
The vegetation data can be retrieved using the Overpass API for OpenStreetMap.
\vspace{0.5cm}

\begin{figure}[H]
    \centering
    \begin{minipage}{0.45\textwidth}
        \centering
        \includegraphics[width=\textwidth]{images/OSM_logo.png}
        \captionsetup{font={scriptsize}}
        \caption{OpenStreetMap \cite{openstreetmap}}
    \end{minipage}
    \begin{minipage}{0.45\textwidth}
        \centering
        \includegraphics[width=\textwidth]{images/OvAPI_logo.png}
        \captionsetup{font={scriptsize}}
        \caption{Overpass API \cite{overpass}}
    \end{minipage}\hfill
\end{figure}

\newpage
Here is an example of a query that retrieves data for all trees within a 50-meter
radius of the coordinates 48.583194, 7.747951, representing the city center of Strasbourg:

\vspace{0.5cm}
\begin{cpp}
// Set the Overpass API endpoint URL
curl_easy_setopt(curl, CURLOPT_URL, "http://overpass-api.de/api/interpreter");
// Set the Overpass query
std::string query = "[out:json];"
                    "node(around:50,48.583194,7.747951)[\"natural\"=\"tree\"];"
                    "out;";
\end{cpp}
\vspace{0.5cm}

The idea is to then store this data in a .json file and the use it to generate
a library.

\vspace{0.5cm}
\begin{lstlisting}[language=json]
{
  "version": 0.6,
  "generator": "Overpass API 0.7.62.1 084b4234",
  "osm3s": {
    "timestamp_osm_base": "2024-03-27T13:32:34Z",
    "copyright": "The data included in this document is from www.openstreetmap.org. The data is made available under ODbL."
  },
  "elements": [
{
  "type": "node",
  "id": 10162019518,
  "lat": 48.5831903,
  "lon": 7.7477189,
  "tags": {
    "circumference": "172.788",
    "diameter_crown": "20",
    "genus": "Platanus",
    "height": "18",
    "natural": "tree",
    "ref": "16279",
    "source": "data.strasbourg.eu - patrimoine_arbore",
    "source:date": "2022-01-02",
    "species": "Platanus acerifolia x"
  }
}

  ]
}
\end{lstlisting}
\vspace{0.5cm}

From the library of trees, we will be able to generate 3D models of trees of
different levels of detail (LOD) thanks to the CGAL library.
CGAL is an open source software project that provides easy access to efficient
and reliable geometric algorithms in the form of a C++ library which is used
in various areas needing geometric computation.\cite{cgal}

\begin{figure}[H]
    \vspace{1.5cm}
    \centering
    \includegraphics[width=0.5\textwidth]{images/cgal_logo.png}
\end{figure}

Softwares like MeshLab, an open source system for processing and editing 3D
triangular meshes\cite{meshlab}, will allow us to visualize the 3D models trees generated with
the help of CGAL.

Starting with LOD0, which represents the least detailed version of the model,
we will progressively complexify the model to LOD1 and LOD2.
Each level adds complexity to the model, making it more
resource-intensive to render.

Here's an example of the different levels of detail for a tree model created
with CGAL:

\begin{figure}[H]
    \vspace{1.5cm}
    \centering
    \includegraphics[width=0.5\textwidth]{images/LOD0.png}
    \captionsetup{font={scriptsize}}
    \caption{LOD0}
\end{figure}

\begin{figure}[H]
    \vspace{1.5cm}
    \centering
    \includegraphics[width=0.5\textwidth]{images/LOD1.png}
    \captionsetup{font={scriptsize}}
    \caption{LOD1}
\end{figure}

\begin{figure}[H]
    \vspace{1.5cm}
    \centering
    \includegraphics[width=0.5\textwidth]{images/LOD2.png}
    \captionsetup{font={scriptsize}}
    \caption{LOD2}
\end{figure}

We will be using the Feel++, an open-source C++ library, to calculate the
shading on buildings. This library is particularly suited for solving
Partial Differential Equations (PDEs), which are essential in the computation
of light and shade on 3D objects\cite{feel++}. By leveraging the capabilities of Feel++,
we can accurately simulate the shading effects on our LOD models of buildings.

The project is supervised by Pierre Alliez
Senior Researcher at Inria center at Côte d'Azur University
and by Vincent Chabannes research engeneer at the University of Strasbourg.
We will have weekly meetings with them to discuss the progress of the project.
The deadlines we have set are the following:
\begin{itemize}
    \item \textbf{V0} : due by March 26, 2024
    \item \textbf{V1} : due by April 23, 2024
    \item \textbf{V2 - Final} : due by May 28, 2024
\end{itemize}

With the help of our supervisors, we have defined the following roadmap:

\begin{figure}[H]
    \centering
    \includegraphics[width=0.5\textwidth]{images/roadmap_v0.png}
    \captionsetup{font={scriptsize}}
    \caption{Roadmap v0}
\end{figure}

\begin{figure}[H]
    \centering
    \includegraphics[width=0.5\textwidth]{images/roadmap_v1.png}
    \captionsetup{font={scriptsize}}
    \caption{Roadmap v1}
\end{figure}

\begin{figure}[H]
    \centering
    \includegraphics[width=0.5\textwidth]{images/roadmap_v2.png}
    \captionsetup{font={scriptsize}}
    \caption{Roadmap v2}
\end{figure}

This will help us to keep track of the progress of the project and to make sure
we are going to deliever the respective versions on time.

This project will be developed in C++.
For the geometric aspect of tree model and meshing we will use the CGAL library.

\newpage
\section{Literature Review}

\newpage

\section{Methodology}

\newpage

\section{Results}

\newpage

\section{Conclusion}

\newpage

\section{References}
\bibliographystyle{plain}
\bibliography{references}

\end{document}