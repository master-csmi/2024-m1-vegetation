\documentclass[10pt]{beamer}
\usepackage{graphicx}
\usepackage{adjustbox}
\usepackage{hyperref}
\usepackage{amsmath}
\usepackage{hyperref}
\usepackage{graphicx}
\usepackage{float}
\usepackage{caption}
\usepackage{listings}
\usepackage{xcolor}
\usepackage{multimedia}


\colorlet{punct}{red!60!black}
\definecolor{background}{RGB}{240, 248, 255}
\definecolor{delim}{RGB}{20,105,176}
\colorlet{numb}{magenta!60!black}

\lstdefinelanguage{json}{
    basicstyle=\ttfamily\footnotesize\color{black},
    numbers=left,
    numberstyle=\scriptsize,
    stepnumber=1,
    numbersep=8pt,
    showstringspaces=false,
    breaklines=true,
    frame=lines,
    backgroundcolor=\color{background},
    literate=
     *{0}{{{\color{numb}0}}}{1}
      {1}{{{\color{numb}1}}}{1}
      {2}{{{\color{numb}2}}}{1}
      {3}{{{\color{numb}3}}}{1}
      {4}{{{\color{numb}4}}}{1}
      {5}{{{\color{numb}5}}}{1}
      {6}{{{\color{numb}6}}}{1}
      {7}{{{\color{numb}7}}}{1}
      {8}{{{\color{numb}8}}}{1}
      {9}{{{\color{numb}9}}}{1}
      {:}{{{\color{punct}{:}}}}{1}
      {,}{{{\color{punct}{,}}}}{1}
      {\{}{{{\color{delim}{\{}}}}{1}
      {\}}{{{\color{delim}{\}}}}}{1}
      {[}{{{\color{delim}{[}}}}{1}
      {]}{{{\color{delim}{]}}}}{1},
}

\lstset{frame=single, showstringspaces=false, columns=fixed, basicstyle={\ttfamily}, commentstyle={\it}, numbers=left, tabsize=4}

\definecolor{codebackground}{RGB}{240, 248, 255}
\definecolor{codecomment}{RGB}{106,153,85}
\definecolor{codekeyword}{RGB}{30,30,255}
\definecolor{codestring}{RGB}{163,21,21}
\definecolor{codenumber}{RGB}{100,100,100}

\lstdefinestyle{modernstyle}{
    backgroundcolor=\color{codebackground},
    commentstyle=\color{codecomment},
    keywordstyle=\color{codekeyword},
    numberstyle=\tiny\color{codenumber},
    stringstyle=\color{codestring},
    basicstyle=\ttfamily\footnotesize\color{black},
    breakatwhitespace=false,
    breaklines=true,
    captionpos=b,
    keepspaces=true,
    numbers=left,
    numbersep=5pt,
    showspaces=false,
    showstringspaces=false,
    showtabs=false,
    tabsize=4
}

\lstset{style=modernstyle}

\usetheme{Copenhagen}
\usecolortheme{default}
\setbeamertemplate{navigation symbols}{}

\title[ExaMA WP1 Vegetation]{
  \includegraphics[width=0.8\textwidth]{images/logo-ufr.png}
  ExaMA WP1 - Vegetation}
\author[Carpi Lapi - Senger]{Giulio CARPI LAPI, Pierre-Antoine SENGER}

\begin{document}

\begin{frame}{Introduction}
  \titlepage
\end{frame}

\begin{frame}{Introduction}
  \Large
  \begin{itemize}
    \Large
    \item Part of the \textbf{HiDALGO2} project
    \item Specifically \textbf{Urban Building Model} use case
    \item Project conducted within \textbf{Cemosis} - \textbf{IRMA}
    \item Supervised by \textbf{Pierre Alliez} and \textbf{Vincent Chabannes}
  \end{itemize}
  \vfill % This adds vertical space to push the images to the bottom
  \centering
  \begin{minipage}{0.3\textwidth}
      \centering
      \includegraphics[width=0.8\textwidth]{images/logo-hidalgo2.png}
  \end{minipage}
  \begin{minipage}{0.3\textwidth}
      \centering
      \includegraphics[width=0.8\textwidth]{images/logo-cemosis.png}
  \end{minipage}
  \begin{minipage}{0.3\textwidth}
      \centering
      \includegraphics[width=0.8\textwidth]{images/logo-irma.png}
  \end{minipage}
\end{frame}

\begin{frame}{Context}
  \begin{figure}
      \centering
      \includegraphics[width=0.8\textwidth]{images/tree-shade.png}
      \captionsetup{font={scriptsize}}
      \caption{Tree providing shade to a building\cite{img:TreeShade}}
  \end{figure}
\end{frame}

\begin{frame}{Context}
  \begin{figure}
      \centering
      \includegraphics[width=\textwidth]{images/heat-street.png}
      \captionsetup{font={scriptsize}}
      \caption{Thermal image of a street depicting heat distribution\cite{img:street_thermography}}
  \end{figure}
\end{frame}

\begin{frame}{Context}
  \Large
  \textbf{Main goals:}
  \begin{itemize}
    \item Integrate \textbf{trees} into \textbf{3D geometric models} of \textbf{urban environments}
    \item Improve the \textbf{accuracy} of \textbf{thermal} and \textbf{energy simulations}
  \end{itemize}
\end{frame}

\begin{frame}{Context: Primiray focus}
  \begin{figure}[H]
    \centering
    \begin{minipage}{0.49\textwidth}
      \centering
      \includegraphics[width=\textwidth]{images/strasbourg-mesh-1.png}
      \captionsetup{font={scriptsize}}
      \caption{Strasbourg 3D model (1)}
    \end{minipage}
    \hfill
    \begin{minipage}{0.49\textwidth}
      \centering
      \includegraphics[width=\textwidth]{images/strasbourg-mesh-2.png}
      \captionsetup{font={scriptsize}}
      \caption{Strasbourg 3D model (2)}
    \end{minipage}
  \end{figure}
\end{frame}  

\begin{frame}{Context: Adaptability}
  \begin{figure}[H]
    \centering
    \begin{minipage}{0.49\textwidth}
      \centering
      \includegraphics[width=\textwidth]{images/mesh-manhattan-1.png}
      \captionsetup{font={scriptsize}}
      \caption{Manhattan 3D model (1)}
    \end{minipage}
    \hfill
    \begin{minipage}{0.49\textwidth}
      \centering
      \includegraphics[width=\textwidth]{images/mesh-manhattan-2.png}
      \captionsetup{font={scriptsize}}
      \caption{Manhattan 3D model (2)}
    \end{minipage}
  \end{figure}
\end{frame}  

\begin{frame}{Objectives}
  \Large
  \begin{itemize}
      \item \textbf{\textcolor{red}{Extracting}} \textbf{tree data} from \textbf{OpenStreetMap}
      \item \textbf{\textcolor{red}{Generating}} \textbf{3D tree models} using \textbf{CGAL}
      \item \textbf{\textcolor{red}{Integrating}} \textbf{tree models} in the \textbf{terrain mesh}
      \item \textbf{\textcolor{red}{Optimizing}} \textbf{computational efficiency}
  \end{itemize}
\end{frame}

\begin{frame}{Sofware and libraries: Overpass API}
  \Large
  \begin{figure}[H]
      \centering
      \includegraphics[width=\textwidth]{images/logo-overpass.png}
  \end{figure}
  \begin{center}
  \Large \textbf{Read-only API} to query data from \textbf{OpenStreetMap}
  \end{center}
\end{frame}

\begin{frame}{Sofware and libraries: OpenStreetMap}
  \Large
  \begin{figure}[H]
      \centering
      \includegraphics[width=0.5\textwidth]{images/logo-openstreetmap.png}
  \end{figure}
  \begin{center}
    \Large Collaborative free \textbf{geographic database}
  \end{center}
\end{frame}

\begin{frame}{Sofware and libraries: Overpass turbo}
    \begin{figure}[H]
      \centering
      \includegraphics[width=\textwidth]{images/ovt-example.png}
      \captionsetup{font={scriptsize}}
      \caption{Query in Overpass turbo interface}
    \end{figure}
\end{frame}

\begin{frame}{Sofware and libraries: Overpass turbo}
  \begin{figure}[H]
    \centering
    \includegraphics[width=0.75\textwidth]{images/ovt-node.png}
  \end{figure}
\end{frame}

\begin{frame}{Sofware and libraries: cURL}
  \Large
  \begin{figure}[H]
      \centering
      \includegraphics[width=0.8\textwidth]{images/logo-curl.png}
  \end{figure}
  \begin{center}
    \Large \textbf{URL} transfer library
  \end{center}
\end{frame}

\begin{frame}[fragile]{Data acquisition: The query}
  \begin{lstlisting}[language=C++]
#include <curl/curl.h>

curl_easy_setopt(curl, CURLOPT_URL,
                "http://overpass-api.de/api/interpreter");

// Set the Overpass query with the bounding box
std::string query =
"[out:json]; (node(" + bbox + ")[\"natural\"=\"tree\"];); out;";

std::cout << "Query: " << query << std::endl;
  \end{lstlisting}
\end{frame}


\begin{frame}[fragile]{Data acquisition: .json output}
\begin{lstlisting}[language=json]
{
  "type": "node",
  "id": 10162018740,
  "lat": 48.5850910,
  "lon": 7.7502624,
  "tags": {
    "circumference": "1.47655",
    "diameter_crown": "5",
    "genus": "Platanus",
    "height": "6",
    "leaf_cycle": "deciduous",
    "leaf_type": "broadleaved",
    "natural": "tree",
    "ref": "16401",
    "source": "data.strasbourg.eu - patrimoine_arbore",
    "source:date": "2022-01-02",
    "species": "Platanus acerifolia x",
    "species:wikidata": "Q24853030"
  }
}
\end{lstlisting}
\end{frame}

\begin{frame}{Data acquisition: Base tree models}
\begin{figure}[H]
    \centering
        \centering
        \includegraphics[width=0.8\textwidth]{images/ginkgo-sketchup.png}
        \captionsetup{font={scriptsize}}
        \caption{Mesh of a Ginkgo tree on Sketchup 3D Warehouse}
\end{figure}
\end{frame}

\begin{frame}{Data acquisition: Base tree models}
  \Large
  \begin{figure}[H]
    \centering
    \begin{minipage}{0.24\textwidth}
        \centering
        \includegraphics[width=\textwidth]{images/tree-abies.png}
        \captionsetup{font={scriptsize}}
        \caption{Abies}
    \end{minipage}\hfill
    \begin{minipage}{0.24\textwidth}
        \centering
        \includegraphics[width=\textwidth]{images/tree-acer.png}
        \captionsetup{font={scriptsize}}
        \caption{Acer}
    \end{minipage}\hfill
    \begin{minipage}{0.24\textwidth}
        \centering
        \includegraphics[width=\textwidth]{images/tree-aesculus.png}
        \captionsetup{font={scriptsize}}
        \caption{Aesculus}
    \end{minipage}\hfill
    \begin{minipage}{0.24\textwidth}
        \centering
        \includegraphics[width=\textwidth]{images/tree-catalpa.png}
        \captionsetup{font={scriptsize}}
        \caption{Catalpa}
    \end{minipage}
\end{figure}

\begin{figure}[H]
    \centering
    \begin{minipage}{0.24\textwidth}
        \centering
        \includegraphics[width=\textwidth]{images/tree-cedrus.png}
        \captionsetup{font={scriptsize}}
        \caption{Cedrus}
    \end{minipage}\hfill
    \begin{minipage}{0.24\textwidth}
        \centering
        \includegraphics[width=\textwidth]{images/tree-liquidanbar.png}
        \captionsetup{font={scriptsize}}
        \caption{Liquidanbar}
    \end{minipage}\hfill
    \begin{minipage}{0.24\textwidth}
        \centering
        \includegraphics[width=\textwidth]{images/tree-platanus.png}
        \captionsetup{font={scriptsize}}
        \caption{Platanus}
    \end{minipage}\hfill
    \begin{minipage}{0.24\textwidth}
        \centering
        \includegraphics[width=\textwidth]{images/tree-quercus.png}
        \captionsetup{font={scriptsize}}
        \caption{Quercus}
    \end{minipage}
\end{figure}
\end{frame}

\begin{frame}[fragile]{Data acquisition: Tree library}
  \begin{lstlisting}[language=json]
  {
      "known_genus": ["Abies",
                      "Acer",
                      "Aesculus",
                          ... ],
      "cedrus_like": [ "Chaemacyparis",
                      "Cupressus",
                          ... ],
      "acer_like": ["Fadus",
                  "Metasequoia",
                  "Sequoiadendron",
                  ... ],
      "liquidambar_like": ["Liriodendron",
                          "Pyrus",
                          "Alnus",
                          ... ],
      "quercus_like": ["Corylus",
                      "Carya",
                      "Fagus",
                      ... ]
  }
  \end{lstlisting}
\end{frame}

\begin{frame}{Sofware and libraries: CGAL}
  \Large
  \begin{figure}[H]
      \centering
      \includegraphics[width=0.8\textwidth]{images/logo-cgal.png}
  \end{figure}
  \begin{center}
    \Large Open source software library for \textbf{computational geometry algorithms}
  \end{center}
\end{frame}

\begin{frame}{Reminder: Delaunay triangulation}
  \Large
  \begin{figure}[H]
    \centering
    \includegraphics[width=0.6\textwidth]{images/delaunay-vectorial.png}
    \captionsetup{font={scriptsize}}
    \caption{Delaunay triangulation.
    The circumcircle of each triangle contains no other point\cite{delaunay-wiki}}
\end{figure}
\end{frame}

\begin{frame}{Reminder: Delaunay and Voronoi}
  \Large
  \begin{figure}[H]
    \centering
    \begin{minipage}{0.49\textwidth}
        \centering
        \includegraphics[width=\textwidth]{images/delaunay-centers.png}
        \captionsetup{font={scriptsize}}
        \caption{Delaunay triangulation with the centers of the circumcircles\cite{delaunay-wiki}}
    \end{minipage}\hfill
    \begin{minipage}{0.49\textwidth}
        \centering
        \includegraphics[width=\textwidth]{images/delaunay-voronoi.png}
        \captionsetup{font={scriptsize}}
        \caption{The dual of the Delaunay triangulation, the Voronoi diagram\cite{delaunay-wiki}}
    \end{minipage}
  \end{figure}
\end{frame}

\begin{frame}{Tree modeling: Alpha Wrapping}
  \begin{figure}[H]
    \centering
        \centering
        \includegraphics[width=\textwidth]{images/alpha-wrapping-bike.jpg}
        \captionsetup{font={scriptsize}}
        \caption{Different LOD of the Alpha Wrapping of a bike\cite{cgal_alpha_wrapper}}
\end{figure}
\end{frame}

\begin{frame}{Tree modeling: Alpha Wrapping}
  \Large
  \textcolor{red}{\textbf{Input:}}
    \begin{itemize}
    \item  3D model with possible defects
    \end{itemize}
    \textcolor{red}{\textbf{Output:} }
    \begin{itemize}
      \item Water-tight mesh
      \item No self-intersections
      \item Strictly enclosing the input
      \item Well shaped triangles
    \end{itemize}
\end{frame}


\begin{frame}{Tree modeling: Alpha Wrapping}
  \Large
  \begin{figure}[H]
    \centering
    \includegraphics[width=\textwidth]{images/alpha-wrapping-ball.jpg}
    \captionsetup{font={scriptsize}}
    \caption{Alpha Wrapping in 2D with Offset and different Alpha parameters}
\end{figure}
\end{frame}


\begin{frame}{Tree modeling: Alpha Wrapping}
  \Large
  \href{https://youtu.be/xIIDolWCrgU}{video link}
  \begin{center}
    \movie[width=1\textwidth,height=0.8\textheight,poster,showcontrols]{}
    {images/alpha-wrapping.mp4}
  \end{center}
\end{frame}

\begin{frame}{Tree modeling: wrapping base tree}
  \Large
  \begin{figure}[H]
    \centering
    \begin{minipage}{0.49\textwidth}
        \centering
        \includegraphics[width=0.8\textwidth]{images/gingko-lod0.png}
        \captionsetup{font={scriptsize}}
        \caption{Ginkgo lod0}
    \end{minipage}\hfill
    \begin{minipage}{0.49\textwidth}
        \centering
        \includegraphics[width=0.8\textwidth]{images/gingko-lod1.png}
        \captionsetup{font={scriptsize}}
        \caption{Ginkgo lod1}
    \end{minipage}
\end{figure}

\begin{figure}[H]
    \centering
    \begin{minipage}{0.49\textwidth}
        \centering
        \includegraphics[width=0.8\textwidth]{images/gingko-lod2.png}
        \captionsetup{font={scriptsize}}
        \caption{Ginkgo lod2}
    \end{minipage}\hfill
    \begin{minipage}{0.49\textwidth}
        \centering
        \includegraphics[width=0.8\textwidth]{images/gingko-lod3.png}
        \captionsetup{font={scriptsize}}
        \caption{Ginkgo lod3}
    \end{minipage}
\end{figure}
\end{frame}

\begin{frame}{Tree modeling: Mercator's projection}
  \begin{figure}[H]
    \centering
    \includegraphics[width=\textwidth]{images/mercator.jpg}
    \captionsetup{font={scriptsize}}
    \caption{Mercator's projection\cite{img:mercator}}
\end{figure}

\end{frame}

\begin{frame}{Tree modeling: Mercator's projection}
  \Large
  A(latitude, longitude) = A($\phi$, $\lambda$),\\
  \begin{equation}
    \text{projection} \Longrightarrow \quad
    \left\{
    \begin{array}{l}
        x =  \lambda - \lambda_{0} \\
        y =  \ln(\tan(\frac{\pi}{4} + \frac{\phi}{2}))
    \end{array}
    \right.
  \end{equation}
  \vfill
  ,where $\lambda_{0}$ is the center of the map
\end{frame}


\begin{frame}{Tree modeling: Mercator's projection}
  \Large
  \begin{figure}[H]
    \centering
    \begin{minipage}{0.49\textwidth}
        \centering
        \includegraphics[width=0.8\textwidth]{images/WGS84-earth-radius.png}
        \captionsetup{font={scriptsize}}
        \caption{Earth as an ellipsoid\cite{mercator-proj}}
    \end{minipage}\hfill
    \begin{minipage}{0.49\textwidth}
        \centering
        \includegraphics[width=0.8\textwidth]{images/WGS84-frame.png}
        \captionsetup{font={scriptsize}}
        \caption{WGS 84 reference frame\cite{mercator-proj}}
    \end{minipage}
\end{figure}

\textit{WGS84toCartesian.hpp} $\Longrightarrow$ \textbf{GPS} to \textbf{Cartesian}
\end{frame}

\begin{frame}{Tree modeling: affine transformation}
  \Large
  \begin{figure}[H]
    \centering
    \includegraphics[width=1\textwidth]{images/ginkgo-bbox.png}
    \captionsetup{font={scriptsize}}
    \caption{Ginkgo tree bounding box}
\end{figure}
\end{frame}

\begin{frame}{Tree modeling: affine transformation}
\begin{figure}[H]
  \centering
  \begin{minipage}{0.49\textwidth}
      \centering
      \includegraphics[width=1\textwidth]{images/mesh-republic-lod3.png}
      \captionsetup{font={scriptsize}}
      \caption{Republic square with LOD 3 trees.}
  \end{minipage}\hfill
  \begin{minipage}{0.49\textwidth}
      \centering
      \includegraphics[width=1\textwidth]{images/ovt-republic.png}
      \captionsetup{font={scriptsize}}
      \caption{Republic square trees from Overpass turbo\cite{overpass-turbo}}
  \end{minipage}
\end{figure}
\end{frame}

\begin{frame}{Tree modeling: affine transformation}
  \Large
  \begin{figure}[H]
    \centering
    \includegraphics[width=1\textwidth]{images/mesh-republic-lod3-side.png}
    \captionsetup{font={scriptsize}}
    \caption{Side view of Republic square with LOD 3 trees}
\end{figure}
\end{frame}

\begin{frame}{Tree modeling: model integration}
  \Large
  \begin{figure}[H]
    \centering
    \includegraphics[width=1\textwidth]{images/mesh-strasbourg-enhanced1.png}
    \captionsetup{font={scriptsize}}
    \caption{Strasbourg 3D model with LOD 0 trees}
\end{figure}
\end{frame}

\begin{frame}{Tree modeling: model integration}
  \Large
  \begin{figure}[H]
    \centering
    \includegraphics[width=1\textwidth]{images/mesh-strasbourg-enhanced2.png}
    \captionsetup{font={scriptsize}}
    \caption{Strasbourg 3D model with LOD 0 trees}
\end{figure}
\end{frame}


\begin{frame}{Benchmark}
  \Large
  \begin{figure}[H]
    \centering
    \includegraphics[width=0.7\textwidth]{images/ovt-bbox1.png}
    \captionsetup{font={scriptsize}}
    \caption{Bounding Box 1: 153.7 m², 12 trees}
\end{figure}
\end{frame}

\begin{frame}{Benchmark}
  \Large
  \begin{figure}[H]
    \centering
    \includegraphics[width=0.7\textwidth]{images/ovt-bbox2.png}
    \captionsetup{font={scriptsize}}
    \caption{Bounding Box 2: 384.0 m², 71 trees}
\end{figure}
\end{frame}

\begin{frame}{Benchmark}
  \Large
  \begin{figure}[H]
    \centering
    \includegraphics[width=0.7\textwidth]{images/ovt-bbox3.png}
    \captionsetup{font={scriptsize}}
    \caption{Bounding Box 3: 626.1 m², 254 trees}
\end{figure}
\end{frame}

\begin{frame}{Benchmark}
  \Large
  \begin{figure}[H]
    \centering
    \includegraphics[width=0.7\textwidth]{images/ovt-bbox4.png}
    \captionsetup{font={scriptsize}}
    \caption{Bounding Box 4: 808.4 m², 513 trees}
\end{figure}
\end{frame}

\begin{frame}{Benchmark: relation LOD-number of faces}
  \Large
  \begin{figure}[H]
    \centering
    \includegraphics[width=1\textwidth]{images/bench-ntree-nfaces.png}
    \captionsetup{font={scriptsize}}
\end{figure}
\end{frame}

\begin{frame}{Benchmark: execution time (Part 1)}
  \Large
  \begin{figure}[H]
    \centering
    \includegraphics[width=1\textwidth]{images/bench-time-ntree-quad.png}
    \captionsetup{font={scriptsize}}
    \caption{\texttt{corefine\_and\_compute\_union}}
  \end{figure}
\end{frame}

\begin{frame}{Benchmark: execution time (Part 2)}
  \Large
  \begin{figure}[H]
    \centering
    \includegraphics[width=1\textwidth]{images/bench-time-ntree-linear.png}
    \captionsetup{font={scriptsize}}
    \caption{\texttt{autorefine\_triangle\_soup}}
  \end{figure}
\end{frame}

\begin{frame}{Prospects}
  \Large
  \begin{itemize}
    \item Account for \textbf{seasonal changes} and \textbf{leaf fall}
    \item \textbf{Solar masks} and \textbf{shading calculations}
  \end{itemize}
\end{frame}

\begin{frame}{Prospects: Leaf fall}
  \Large
  \begin{itemize}
    \item Seasonal leaf changes
    \item \textbf{More} leaves in \textbf{spring}/\textbf{summer} and \textbf{fewer} in \textbf{fall}/\textbf{winter}
  \end{itemize}
\end{frame}

\begin{frame}{Prospects: Shading calculations}
  \Large
  \begin{itemize}
    \item Ray tracing and shading simulations with \textbf{Feel++}
    \item \textbf{Impact} of \textbf{trees} in urban \textbf{microclimates}
  \end{itemize}
\end{frame}

\begin{frame}{Prospects: Shading calculations}
  \begin{figure}
      \centering
      \includegraphics[width=0.65\textwidth]{images/shading-1.png}
      \captionsetup{font={scriptsize}}
      \caption{Shading calculations in urban environments (1)}
  \end{figure}
\end{frame}

\begin{frame}{Prospects: Shading calculations}
  \begin{figure}
      \centering
      \includegraphics[width=0.65\textwidth]{images/shading-2.png}
      \captionsetup{font={scriptsize}}
      \caption{Shading calculations in urban environments (2)}
  \end{figure}
\end{frame}

\begin{frame}{Prospects: Shading calculations}
  \begin{figure}
      \centering
      \includegraphics[width=0.65\textwidth]{images/shading-3.png}
      \captionsetup{font={scriptsize}}
      \caption{Shading calculations in urban environments (3)}
  \end{figure}
\end{frame}

\begin{frame}{Prospects: Shading calculations}
  \begin{figure}
      \centering
      \includegraphics[width=0.65\textwidth]{images/shading-4.png}
      \captionsetup{font={scriptsize}}
      \caption{Shading calculations in urban environments (4)}
  \end{figure}
\end{frame}

\begin{frame}{Prospects: Shading calculations}
  \begin{figure}
      \centering
      \includegraphics[width=0.65\textwidth]{images/shading-5.png}
      \captionsetup{font={scriptsize}}
      \caption{Shading calculations in urban environments (5)}
  \end{figure}
\end{frame}

\begin{frame}{Prospects: Shading calculations}
  \begin{figure}
      \centering
      \includegraphics[width=0.65\textwidth]{images/shading-6.png}
      \captionsetup{font={scriptsize}}
      \caption{Shading calculations in urban environments (6)}
  \end{figure}
\end{frame}

\begin{frame}{Prospects: Shading calculations}
  \begin{figure}
      \centering
      \includegraphics[width=0.65\textwidth]{images/shading-7.png}
      \captionsetup{font={scriptsize}}
      \caption{Shading calculations in urban environments (7)}
  \end{figure}
\end{frame}

\begin{frame}{Prospects: Shading calculations}
  \begin{figure}
      \centering
      \includegraphics[width=0.65\textwidth]{images/shading-8.png}
      \captionsetup{font={scriptsize}}
      \caption{Shading calculations in urban environments (8)}
  \end{figure}
\end{frame}

\begin{frame}{Prospects: Shading calculations}
  \begin{figure}
      \centering
      \includegraphics[width=0.65\textwidth]{images/shading-9.png}
      \captionsetup{font={scriptsize}}
      \caption{Shading calculations in urban environments (9)}
  \end{figure}
\end{frame}

\begin{frame}{Prospects: Shading calculations}
  \begin{figure}
      \centering
      \includegraphics[width=0.65\textwidth]{images/shading-10.png}
      \captionsetup{font={scriptsize}}
      \caption{Shading calculations in urban environments (10)}
  \end{figure}
\end{frame}

\begin{frame}{Prospects: Shading calculations}
  \begin{figure}
      \centering
      \includegraphics[width=0.65\textwidth]{images/shading-11.png}
      \captionsetup{font={scriptsize}}
      \caption{Shading calculations in urban environments (11)}
  \end{figure}
\end{frame}

\begin{frame}{Prospects: Shading calculations}
  \begin{figure}
      \centering
      \includegraphics[width=0.65\textwidth]{images/shading-12.png}
      \captionsetup{font={scriptsize}}
      \caption{Shading calculations in urban environments (12)}
  \end{figure}
\end{frame}

\begin{frame}{Prospects: Shading calculations}
  \begin{figure}
      \centering
      \includegraphics[width=0.65\textwidth]{images/shading-13.png}
      \captionsetup{font={scriptsize}}
      \caption{Shading calculations in urban environments (13)}
  \end{figure}
\end{frame}

\begin{frame}{Conclusion}
  \Large
  \begin{itemize}
    \item \textbf{Tree data} extraction from \textbf{OpenStreetMap}
    \item \textbf{3D tree models} generation with \textbf{CGAL}
    \item \textbf{Integration} of tree models in the \textbf{terrain mesh}
    \item \textbf{Optimization} of \textbf{computational efficiency}
  \end{itemize}
\end{frame}

\begin{frame}{Conclusion}
  \Large
  \textbf{ExaMA WP1 - Vegetation}:
  \begin{itemize}
    \item Foundation for future urban energy simulations
    integrating vegetation into the models
  \end{itemize}
\end{frame}

\begin{frame}{The end}
  \Large
  \centering
  \textbf{Thank you for your attention!}
\end{frame}

\nocite{*}
\bibliographystyle{unsrt}
\bibliography{references}

\end{document}